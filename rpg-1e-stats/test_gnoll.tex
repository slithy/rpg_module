\documentclass[a4paper,sansserif,1e]{rpg-module}

\begin{document}

\title{\textrm{\LaTeX}~RPG Module Class: 1e Stats Test}

\part{1e Stats Test}

The default stat blocks are the full (long) version. Using \textbf{gnoll} as an example, here are all the ways the stat
blocks can be displayed:

\section{Long statblock format}

The default is the long format, here shown as a statblock for a single monster and a statblock for multiple monsters (to
test that single/plural handling works OK):

\statblock{gnoll}{1}{16}
\statblock{gnoll}{5}{16,14,12,9,8}

\noindent The stats can also be included inline in a paragraph of text like this:
You find yourself face to face with the \stats[Gnoll Chieftain!]{gnoll}{1}{16}.

\section{Short statblock format}

The \verb|\longstats| and \verb|\shortstats| commands display the stat blocks in long/short format. It is possible to
mix both formats in the same document (e.g. use long format on first appearance and short format for subsequent mentions
of monsters of the same type).

The \verb|\stats| command is an alias to \verb|\longstats|, but it can be redefined as \verb|\shortstats|. Here are the
same stats as above in short format:

\renewcommand{\stats}{\shortstats}

\statblock{gnoll}{1}{16}
\statblock{gnoll}{5}{16,14,12,9,8}

\noindent If you want to display stats inline in brackets, that's also possible: You find yourself face to face with the
fearsome Gnoll Chieftain! \stats[(\ignorespaces]{gnoll}{1}{16}).

\section{New Monster format}

The format for listing new monsters at the end of a module is shown below. Optionally, include an image file with the listing.

\begin{newmonster}{gnoll}
Gnolls travel and live in rapacious bands of loose organization, with the
largest dominating the rest. These bands recognize no other gnoll as
supreme, but they do not necessarily dislike other bands, and on occasion
two or more such groups will ioin together briefly in order to fight, raid,
loot, or similarly have greater chance of success against some common foe
or potential victim. They are adaptable and inhabit nearly any area save
those which are arctic and/or arid. They hove a so-called king, very
powerful personally and with a double normal-sized following, but his
authority extends only as far as his reach.
\end{newmonster}

\begin{newmonster}[gnoll.png]{gnoll}
Gnolls travel and live in rapacious bands of loose organization, with the
largest dominating the rest. These bands recognize no other gnoll as
supreme, but they do not necessarily dislike other bands, and on occasion
two or more such groups will ioin together briefly in order to fight, raid,
loot, or similarly have greater chance of success against some common foe
or potential victim. They are adaptable and inhabit nearly any area save
those which are arctic and/or arid. They hove a so-called king, very
powerful personally and with a double normal-sized following, but his
authority extends only as far as his reach.
\end{newmonster}

Test 2-column and 3-column new monster formats:

\begin{newmonster2}{gnoll}{gnoll}
Gnolls travel and live in rapacious bands of loose organization, with the
largest dominating the rest. These bands recognize no other gnoll as
supreme, but they do not necessarily dislike other bands, and on occasion
two or more such groups will ioin together briefly in order to fight, raid,
loot, or similarly have greater chance of success against some common foe
or potential victim. They are adaptable and inhabit nearly any area save
those which are arctic and/or arid. They hove a so-called king, very
powerful personally and with a double normal-sized following, but his
authority extends only as far as his reach.
\end{newmonster2}

\begin{newmonster3}{gnoll}{gnoll}{gnoll}
Gnolls travel and live in rapacious bands of loose organization, with the
largest dominating the rest. These bands recognize no other gnoll as
supreme, but they do not necessarily dislike other bands, and on occasion
two or more such groups will ioin together briefly in order to fight, raid,
loot, or similarly have greater chance of success against some common foe
or potential victim. They are adaptable and inhabit nearly any area save
those which are arctic and/or arid. They hove a so-called king, very
powerful personally and with a double normal-sized following, but his
authority extends only as far as his reach.
\end{newmonster3}

\onecolumn

\section{Wandering Monster Table}

\begin{wanderingmonsters}[b]
\wanderitem{gnoll}{} % default number appearing
\wanderitem{gnoll}{1--8} % specify number appearing
\end{wanderingmonsters}

\section{Monster Roster Table}

\begin{monsterroster}
\rosteritem{Room 1}{gnoll}{6}{9 each}
\end{monsterroster}

\end{document}
